\fancyhf{}
\fancyfoot[C]{\thepage}


\chapter{PENDAHULUAN}

\section{\uppercase{LATAR BELAKANG}}


\par Saat ini, penggunaan \textit{Global Positioning System} (GPS) telah menjadi vulgar dan ada di sekitar kita dalam kehidupan sehari-hari. Dengan menggunakan jaringan satelit global, pengguna atau objek dapat ditemukan dengan setidaknya empat satelit berbeda melalui proses trilaterasi. Teknologi ini menjadi sangat andal dan akurat, tetapi tidak berfungsi di dalam gedung atau di bawah tanah. Hal ini disebabkan hilangnya sinyal satelit karena harus melalui struktur padat, seperti dinding. Untuk lebih meningkatkan kerugian, bangunan dengan struktur logam di fondasinya akan melihat peningkatan penurunan kekuatan sinyal atau bahkan kerugian total karena efek sangkar Faraday. Masalah ini dapat memengaruhi aplikasi potensial untuk lokasi dalam ruangan seperti melacak mobil otomatis di dalam gudang, menemukan pasien dan pengunjung di fasilitas kesehatan, melacak inventaris di kantor pintar, di antara aplikasi lainnya \citep{Santos2021}

\par Dalam hal ini, \textit{crowdsourcing} adalah solusi yang menjanjikan yang mengurangi biaya survei dan pemeliharaan ini. Dimungkinkan untuk membuat peta radio dengan menyebarkan tugas pengumpulan data ke beberapa pengguna, dan menggabungkan sidik jari yang dikumpulkan dengan tepat. Meskipun pendekatan \textit{crowdsourcing} sebagian besar telah mengatasi biaya tenaga kerja yang intensif untuk survei lokasi, kelemahannya tetap ada.  Dalam situasi dinamis yang lebih realistis, pengguna bergerak dan sampel terbatas dikumpulkan di setiap lokasi, yang menyebabkan kesalahan signifikan saat mencocokkan dengan pengamatan RSS yang dikumpulkan selama fase survei lokasi \citep{Sun2019}.

\par Tantangan lain yang dihadapi sistem lokalisasi berbasis \textit{ } saat ini adalah masalah keandalan dan akurasi lokalisasi. Hal ini terutama karena variasi kekuatan sinyal dalam waktu yang disebabkan oleh propagasi multipath di lingkungan dalam ruangan. Perubahan karakteristik multipath dapat muncul dari, misalnya, pergerakan orang dan furnitur dan pembukaan dan penutupan pintu. Ini menghasilkan perubahan pada jalur pantulan dan difraksi sinyal, yang menyebabkan penurunan kinerja lokalisasi. Karena propagasi multipath di lingkungan dalam ruangan tidak dapat dihindari, kami mengusulkan untuk menguranginya dengan memanfaatkan informasi sebelumnya tambahan yang tersedia untuk mengurangi ketidakpastian RSS untuk pelokalan \citep{Sun2019}.

\par Penentuan posisi dalam ruangan dan estimasi lokasi di dalam gedung masih menjadi tantangan di platform \textit{Internet of Things} (IoT). Dalam beberapa tahun terakhir, objek \textit{cyber} dan fisik telah terintegrasi karena peningkatan jumlah perangkat penginderaan dan jaringan nirkabel di sekitar lingkungan hidup kita. Kombinasi yang disebutkan di atas telah menghasilkan asal-usul konsep baru yang diselidiki sebagai \textit{Internet of Things}. Di antara konsep yang muncul dalam konteks IoT, \textit{crowdsourcing} adalah salah satu blok bangunan yang paling penting, yang merupakan titik persimpangan berbagai hal dan teknik berbasis manusia. Solusi berbasis manusia terutama didasarkan pada kebijaksanaan orang banyak dan fakta bahwa keputusan yang dibuat oleh sekelompok orang mengarah pada hasil yang luar biasa dibandingkan dengan keputusan yang dibuat oleh individu \citep{Sun2019}.

\par \textit{Crowdsourcing} adalah cara memperoleh layanan, ide, dan data berharga dari sekelompok orang, crowdsensing adalah prinsip yang sama hanya saja data diperoleh oleh perangkat atau sensor dan bukan dari input manusia. Kontribusi pengguna adalah faktor terpenting dari sistem crowdsensing mana pun yang memfasilitasi keandalan, cakupan area penginderaan, dan kualitas data. Crowdsensing seluler mengacu pada kemampuan untuk terus bergerak dan merasakan sejumlah data  besar dengan smartphone. Crowdsensing menawarkan peluang yang belum pernah terjadi sebelumnya untuk memanfaatkan sensor inersia perangkat seluler karena mobilitas manusia dapat memberikan cakupan penginderaan yang tak ada bandingannya dan transmisi data tanpa biaya jaringan sensor tradisional yang tidak layak.

\par Hal di atas kemudian melatar belakangi penelitian ini. Penelitian ini akan membangun sebuah aplikasi yang bertujuan untuk mengetahui posisi suatu individu atau pengguna aplikasi yang berada di dalam gedung untuk melakukan estimasi jumlah orang di dalam gedung berbasis \textit{Crowsourcing Indoor Localization System}. \textit{Crowsourcing Indoor Localization System} merupakan   sebuah   teknologi   yang   dapat digunakan  untuk  menentukan  posisi  seseorang  di  dalam  ruangan  atau  bangunan dengan lebih akurat.

\par Pada penelitian ini, mengandalkan teknologi  \textit{Bluetooth Low Energy} (BLE).  BLE \textit{Beacon} pada dasarnya adalah sebuah perangkat yang sangat sederhana berupa perangkat wireless kecil yang berbasiskan \textit{Bluetooth Low Energy} yang mentransmisikan sinyal radio secara terus menerus yang berkaitan dengan ID dari \textit{beacon} tersebut. Dengan menggunakan \textit{Smartphone Android} terkini, BLE sangat mudah untuk dibaca dan dideteksi. Beberapa informasi yang diperoleh pada pembacaan ini, seperti data sensor dan estimasi jarak antara \textit{beacon} dengan Smartphone. Hanya dengan kedua data tersebut, developer dapat berkreasi untuk mengembangkan banyak aplikasi yang unik, aplikatif, dan dapat bermanfaat untuk optimasi sistem di industri juga manfaat lainnya.

\par Pada proses mapping atau pengumpulan dataset digunakan sebanyak 20 alat transmisi BLE yang disebut \textit{beacon}. Masing-masing alat BLE atau \textit{beacon} diletakkan di berbagai tempat di gedung Blok A Fakultas Matematika dan Ilmu Pengetahuan Alam Universitas Syiah Kuala, diantaranya yaitu  5 \textit{Beacon} diletakkan pada lantai 1, 2 \textit{Beacon} diletakkan pada tangga dari lantai 1 ke lantai 2,  2 \textit{Beacon} diletakkan pada lantai 2,  2 \textit{Beacon} diletakkan pada tangga dari lantai 2 ke lantai 3 dan 9 \textit{Beacon} diletakkan pada lantai 3. Data yang dikumpulkan  tersebut  berbentuk  numerik  berkisar  antara  0  sampai  dengan  -110, dimana semakin nilai dari RSSI mendekati 0, maka pengguna semakin dekat dengan alat \textit{Beacon}  pemancar,  sedangkan  jika  nilai  RSSI  yang  didapat  lebih  dari  sama dengan -110 maka pengguna dianggap di luar dari jangkauan alat \textit{Beacon} pemancar.

\par

\fancyhf{}
\fancyfoot[R]{\thepage}

\section{\uppercase{RUMUSAN MASALAH}}
Berdasarkan latar belakang yang telah diuraikan, terdapat beberapa rumusan masalah pada penelitian ini sebagai berikut:
\begin{enumerate}
	\item Bagaimana mengimplementasikan \textit{Crowdsourcing Indoor Localization System} berbasis BLE untuk proses penentuan lokasi dan estimasi orang dalam gedung.
	\item Bagaimana tingkat akurasi prediksi setiap lokasi pengguna saat melakukan proses  penentuan lokasi dalam gedung menggunakan \textit{Crowdsourcing Indoor Localization System} berbasis BLE.
	      %\item Bagaimana tingkat akurasi prediksi lokasi pengguna yang berada di gedung A FMIPA Unsyiah dengan menggunakan \textit{Indoor Localization System} berbasis BLE.
	\item Bagaimana tingkat akurasi prediksi setiap lokasi saat melakukan proses estimasi orang dalam gedung menggunakan \textit{Crowdsourcing Indoor Localization System} berbasis BLE.

\end{enumerate}

\section{\uppercase{TUJUAN PENELITIAN}}
Berdasarkan rumusan masalah yang telah disebutkan sebelumnya, maka dapat dipaparkan tujuan dari tugas akhir ini adalah sebagai berikut:
\begin{enumerate}
	\item Mengimplementasikan cara kerja layanan \textit{Crowdsourcing Indoor Localization System} berbasis BLE untuk proses penentuan lokasi dan estimasi orang dalam gedung menggunakan aplikasi berbasis Android.
	\item Mengimplementasikan algoritma \textit{\textit{Support Vector Machine}} (SVM) untuk klasifikasi penentuan lokasi setiap pengguna dan estimasi orang dalam gedung.
	\item Menganalisis keakuratan \textit{reference point} yang dipetakan secara urut pada proses survei penelitian berdasarkan jumlah \textit{Beacon} yang digunakan.
	\item Menganalisa fungsionalitas aplikasi menggunakan metode \textit{Black Box Testing} dan menganalisa usability aplikasi menggunakan metode \textit{System Usability Scale} (SUS).
\end{enumerate}


\section{\uppercase{MANFAAT PENELITIAN}}
Adapun manfaat dari penelitian ini adalah sebagai berikut:
\begin{enumerate}
	\item Memberikan kemudahan untuk pengguna yang baru datang ke gedung FMIPA USK, guna membantu penentuan lokasi dalam gedung.
	\item Memberikan kemudahan untuk menganalisis data estimasi orang dalam gedung untuk membantu pencegahan penyebaran Covid-19.

\end{enumerate}


% Baris ini digunakan untuk membantu dalam melakukan sitasi
% Karena diapit dengan comment, maka baris ini akan diabaikan
% oleh compiler LaTeX.
\begin{comment}
\bibliography{daftar-pustaka}
\end{comment}
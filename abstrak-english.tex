\begin{abstracteng}

    \textit{Global Positioning System is a very reliable technology in determining the user's location, but cannot be used if it is in a building or underground. Then the solution is Indoor Localization System (ILO). Indoor Localization System is a system that can determine the location of the smartphone owned by each individual in the room. Then there is no application to analyze the estimated number of people in the building. Based on these problems, this research was conducted using the Support Vector Machine (SVM) algorithm. The data used in this study is signal data as much as 2048 data which is divided into 6 classes. This research class represents 6 types of locations in building A FMIPA USK, namely Stairs 1, Stairs 2, Floor 1, GIS Lab Corridor, Database Lab Corridor, and Network Lab Corridor. In this study, an android-based software system and web service flask application were built with the application of the SVM algorithm to build an application to determine the location and estimate the number of people in the building. The final result of this study resulted in an SVM algorithm accuracy of 94\% with an average application test result using the Usability Metric for User Experience UMUX method of 92.39\%. The accuracy test at each location was carried out 10 times for each location to get an accuracy value of 86.67\%.}


    \bigskip
    \noindent
    \textbf{Kata kunci :} \textit{Indoor Localization System}, \textit{Web Service Flask}, \textit{Support Vector Machine}, \textit{Usability Metric for User Experience}.
\end{abstracteng}
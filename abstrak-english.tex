\begin{abstracteng}

    \textit{The Global Positioning System is a very reliable technology in
        determines the user's location, but cannot be used indoors or underground. This deficiency can be overcome by the Indoor Localization System (ILO), which is a system for determining the location of smartphones owned by each individual in the room so that it can be used to estimate the number of people in the building. Therefore, this study utilizes the Support Vector Machine (SVM) algorithm in determining the location and estimating the number of people in the building using Bluetooth Low Energy as one of the technologies of the Indoor Localization System. The data used in this study is signal data as much as 2048 data which is divided into 6 classes. The class in question represents 6 classes of locations in building A FMIPA USK, namely Stair 1, Stair 2, Floor 1, GIS Lab Corridor, Database Lab Corridor, and Network Lab Corridor. In this study, an Android-based application and a web service application were built that apply the SVM algorithm in determining the location and estimating the number of people in the building. The application of the SVM algorithm in determining the user's location results in an accuracy of testing data of 95\%. Android-based applications were also tested for usability using the UMUX Usability Metric for User Experience method which resulted in a score of 93.84 while the UMUX-lite score was 84.13. Both scores are in the A+ category. This means that the application is well received, and has a B or Excellence scale. In addition, the actual accuracy test at each location was also carried out 20 times per location and obtained an accuracy value of 92,5\%.}


    \bigskip
    \noindent
    \textbf{Kata kunci :} Android, \textit{Indoor Localization System}, \textit{Web Service}, \textit{Support Vector Machine}, \textit{Usability Metric for User Experience}.
\end{abstracteng}
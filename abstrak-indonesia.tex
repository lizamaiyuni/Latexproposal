\begin{abstractind}
    \textit{Global Positioning System} adalah teknologi yang sangat andal dalam menentukan lokasi pengguna, namun tidak bisa digunakan jika di dalam gedung atau bawah tanah. Maka solusinya adalah \textit{Indoor Localization System} (ILO). \textit{Indoor Localization System} merupakan suatu sistem yang dapat menentukan lokasi \textit{smartphone} yang dimiliki setiap individu di dalam ruangan. Kemudian belum adanya aplikasi untuk menganalisis estimasi jumlah orang di dalam gedung. Berdasarkan permasalahan tersebut, penelitian ini dilakukan menggunakan algoritma \textit{Support Vector Machine} (SVM). Data yang digunakan pada penelitian ini adalah data sinyal sebanyak 2048 data yang terbagi menjadi 6 kelas. Kelas penelitian ini mewakilkan 6 jenis lokasi di gedung A FMIPA USK yaitu Tangga 1, Tangga 2, Lantai 1, Koridor Lab GIS, Koridor Lab Database, dan Koridor Lab Jaringan. Pada penelitian ini juga, dibangun sistem perangkat lunak berbasis android dan aplikasi \textit{web service flask} dengan penerapan algoritma SVM untuk membangun aplikasi penentuan lokasi dan estimasi jumlah orang di dalam gedung. Hasil akhir dari penelitian ini menghasilkan akurasi algoritma SVM sebesar 94\% dengan hasil rata-rata pengujian aplikasi menggunakan metode \textit{Usability Metric for User Experience} UMUX sebesar 92,39\%. Pengujian keakuratan pada setiap lokasi yang dilakukan sebanyak 10 kali tiap lokasi mendapatkan nilai akurasi sebesar 86,67\%.

    \bigskip
    \noindent
    \textbf{Kata kunci :} \textit{Indoor Localization System}, \textit{Web Service Flask}, \textit{Support Vector Machine}, \textit{Usability Metric for User Experience}.
\end{abstractind}
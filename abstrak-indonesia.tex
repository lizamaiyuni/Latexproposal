\begin{abstractind}
	\textit{Global Positioning System} adalah teknologi yang sangat andal dalam menentukan lokasi pengguna, namun tidak bisa digunakan di dalam gedung atau bawah tanah. Kekurangan tersebut dapat diatasi oleh \textit{Indoor Localization System} (ILO) yang merupakan suatu sistem penentuan lokasi \textit{smartphone} yang dimiliki setiap individu di dalam ruangan sehingga dapat digunakan untuk melakukan estimasi jumlah orang di dalam gedung. Oleh karena itu, penelitian ini memanfaatkan algoritma \textit{Support Vector Machine} (SVM) dalam penentuan lokasi dan estimasi jumlah orang di dalam Gedung menggunakan Bluetooth Low Energy sebagai salah satu teknologi dari Indoor Localization System. Data yang digunakan pada penelitian ini adalah data sinyal sebanyak 2048 data yang terbagi menjadi 6 kelas. Kelas yang dimaksud mewakilkan 6 jenis lokasi di gedung A FMIPA USK yaitu Tangga 1, Tangga 2, Lantai 1, Koridor Lab GIS, Koridor Lab Database, dan Koridor Lab Jaringan. Pada penelitian ini juga, dibangun aplikasi berbasis Android dan aplikasi \textit{web service} yang menerapkan algoritma SVM dalam penentuan lokasi dan estimasi jumlah orang di dalam gedung. Penerapan dari algoritma SVM dalam menentukan lokasi pengguna menghasilkan akurasi data testing sebesar 95\%. Aplikasi berbasis Android juga diuji usabilitynya dengan metode \textit{Usability Metric for User Experience} UMUX yang menghasilkan nilai  sebesar 93,84 sedangkan skor UMUX-lite adalah 84,13. Kedua skor tersebut berada pada kategori A+. Hal ini bermakna aplikasi dapat diterima dengan baik, dan memiliki skala B atau \textit{Excellence} Selain itu, pengujian keakuratan secara nyata  pada setiap lokasi juga dilakukan sebanyak 10 kali per lokasi dan mendapatkan nilai akurasi sebesar 86,67\%.

	\bigskip
	\noindent
	\textbf{Kata kunci :} \textit{Indoor Localization System}, \textit{Web Service}, Android, \textit{Support Vector Machine}, \textit{Usability Metric for User Experience}.
\end{abstractind}
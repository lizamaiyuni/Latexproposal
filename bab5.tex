%-------------------------------------------------------------------------------
%                            	BAB V
%               		KESIMPULAN DAN SARAN
%-------------------------------------------------------------------------------
\fancyhf{} 
\fancyfoot[C]{\thepage}
\chapter{KESIMPULAN DAN SARAN}

\section{\uppercase{KESIMPULAN}}
	Berdasarkan penelitian yang telah dilakukan, dapat diambil kesimpulan bahwa:
	\begin{enumerate}
		\item \textit{Indoor Positioning System} telah berhasil diimplementasikan dengan menggunakan metode klasifikasi K-NN untuk memprediksi lokasi pengguna di dalam ruangan atau gedung (\textit{indoor}).
		\item Metode klasifikasi K-NN dengan parameter pengujian nilai K=5 untuk data kekuatan sinyal pada jenis \textit{reference point} urut, memiliki \textit{F-Measure} paling baik dibandingkan dengan parameter pengujian lainnya dengan nilai 78,60\%.
		\item Berdasarkan hasil pengujian akurasi klasifikasi dengan metode K-NN menggunakan 6 Beacon dengan parameter pengujian nilai K=5 untuk jenis \textit{referencen point} acak, memiliki F-Measure paling baik dengan nilai 96,2\% dibandingkan dengan menggunakan 3 Beacon.
		\item Berdasarkan hasil pengujian usabilitas menggunakan metode SUS, Aplikasi Kehadiran Dosen dan Aplikasi Kehadiran Mahasiswa dapat diterima dan mudah digunakan dilihat dari tingkat pemahaman pengguna.
		\item Berdasarkan hasil pengujian fungsionalitas menggunakan \textit{Black Box}, Aplikasi Kehadiran Dosen dan Aplikasi Kehadiran Mahasiswa telah berjalan sesuai dengan alur bisnis.
	\end{enumerate}



\section{\uppercase{SARAN}}

	Penelitian ini masih banyak kekurangan sehingga perlu dikembangkan agar menjadi lebih baik. Berikut adalah saran untuk penelitian ini:
	\begin{enumerate}
		\item Tampilan dari aplikasi berbasis Android dibuat lebih \textit{user-friendly}.
		\item Menambahkan fitur-fitur yang lebih banyak lagi pada Aplikasi Kehadiran Dosen seperti: dapat menghapus mahasiswa yang tidak benar-benar berada di dalam ruang kuliah dan memberhentikan proses pencatatan kehadiran saat materi kuliah sudah habis (sebelum jam mata kuliah berakhir).
		\item Aplikasi Kehadiran Dosen dan Aplikasi Kehadiran Mahasiswa hanya bisa berjalan pada \textit{smartphone} Android versi diatas 8.0 dikarenakan menggunakan \textit{Foreground Service} bawaan Android. Sebaiknya ditemukan metode yang lebih baik lagi agar kedua aplikasi tersebut bisa berjalan pada \textit{smartphone} Android versi dibawah 8.0.
		\item Aplikasi Kehadiran Dosen dan Aplikasi Kehadiran Mahasiswa baiknya dibangun juga  versi iOS.
		\item Mencari metode klasifikasi atau metode penentuan lokasi yang lebih baik lagi untuk memprediksi lokasi pengguna.
		\item Sebaiknya ditambahkan lagi jumlah Beacon disetiap ruangan supaya meningkatkan keakuratan klasifikasi.
		\item Pengoptimisasian algoritma perhitungan jarak juga dibutuhkan, guna mengurangi waktu komputasi saat melakukan proses absen apabila kelas yang digunakan sudah sangat banyak.
	\end{enumerate}

\fancyhf{} 
\fancyfoot[R]{\thepage}
% Baris ini digunakan untuk membantu dalam melakukan sitasi
% Karena diapit dengan comment, maka baris ini akan diabaikan
% oleh compiler LaTeX.

\begin{comment}
\bibliography{daftar-pustaka}
\end{comment}

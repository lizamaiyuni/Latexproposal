%-------------------------------------------------------------------------------
%                            	BAB V
%               		KESIMPULAN DAN SARAN
%-------------------------------------------------------------------------------
\fancyhf{}
\fancyfoot[C]{\thepage}
\chapter{KESIMPULAN DAN SARAN}

\section{\uppercase{KESIMPULAN}}
Berdasarkan penelitian yang telah dilakukan, dapat diambil kesimpulan bahwa:
\begin{enumerate}
	\item \textit{Indoor Localization System} telah berhasil diimplementasikan dengan menggunakan metode klasifikasi SVM untuk memprediksi lokasi pengguna di dalam gedung A FMIPA USK.
	\item Metode klasifikasi SVM memiliki F-Measure paling
	      baik dibandingkan dengan parameter pengujian lainnya dengan nilai ......
	\item Berdasarkan hasil pengujian akurasi klasifikasi dengan metode SVM menggunakan 31 Beacon memiliki F-Measure paling baik dengan nilai  ......
	\item Berdasarkan hasil pengujian usabilitas menggunakan metode UMUX, Aplikasi Penentuan Lokasi dan Estimasi Jumlah Orang dapat diterima dan mudah digunakan dilihat dari tingkat pemahaman pengguna.
	\item Berdasarkan hasil pengujian fungsionalitas menggunakan \textit{Black Box}, Aplikasi Penentuan Lokasi dan Estimasi Jumlah Orang telah berjalan sesuai dengan alur bisnis.
\end{enumerate}



\section{\uppercase{SARAN}}

Penelitian ini masih banyak kekurangan sehingga perlu dikembangkan agar menjadi lebih baik. Berikut adalah saran untuk penelitian ini:
\begin{enumerate}
	\item Tampilan dari aplikasi berbasis Android dibuat lebih \textit{user-friendly}.
	\item Menambahkan fitur-fitur yang lebih banyak lagi pada Aplikasi Kehadiran Dosen seperti: dapat menghapus mahasiswa yang tidak benar-benar berada di dalam ruang kuliah dan memberhentikan proses pencatatan kehadiran saat materi kuliah sudah habis (sebelum jam mata kuliah berakhir).
	\item Aplikasi Kehadiran Dosen dan Aplikasi Kehadiran Mahasiswa hanya bisa berjalan pada \textit{smartphone} Android versi diatas 8.0 dikarenakan menggunakan \textit{Foreground Service} bawaan Android. Sebaiknya ditemukan metode yang lebih baik lagi agar kedua aplikasi tersebut bisa berjalan pada \textit{smartphone} Android versi dibawah 8.0.
	\item Aplikasi Kehadiran Dosen dan Aplikasi Kehadiran Mahasiswa baiknya dibangun juga  versi iOS.
	\item Mencari metode klasifikasi atau metode penentuan lokasi yang lebih baik lagi untuk memprediksi lokasi pengguna.
	\item Sebaiknya ditambahkan lagi jumlah Beacon disetiap ruangan supaya meningkatkan keakuratan klasifikasi.
	\item Pengoptimisasian algoritma perhitungan jarak juga dibutuhkan, guna mengurangi waktu komputasi saat melakukan proses absen apabila kelas yang digunakan sudah sangat banyak.
\end{enumerate}

\fancyhf{}
\fancyfoot[R]{\thepage}
% Baris ini digunakan untuk membantu dalam melakukan sitasi
% Karena diapit dengan comment, maka baris ini akan diabaikan
% oleh compiler LaTeX.

\begin{comment}
\bibliography{daftar-pustaka}
\end{comment}

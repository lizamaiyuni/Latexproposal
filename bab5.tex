%-------------------------------------------------------------------------------
%                            	BAB V
%               		KESIMPULAN DAN SARAN
%-------------------------------------------------------------------------------
\fancyhf{}
\fancyfoot[C]{\thepage}
\chapter{KESIMPULAN DAN SARAN}

\section{\uppercase{KESIMPULAN}}
Berdasarkan penelitian yang telah dilakukan, dapat diambil kesimpulan bahwa:
\begin{enumerate}
	\item Cara kerja layanan \textit{Crowdsourcing Indoor Localization System} berbasis BLE untuk proses penentuan lokasi dan estimasi orang di dalam gedung menggunakan aplikasi berbasis Android telah berhasil diimplementasikan dengan memanfaatkan algoritma klasifikasi SVM.
	\item Metode klasifikasi SVM untuk data kekuatan sinyal pada jenis \textit{reference point} urut telah berhasil diimplementasikan dengan memiliki rata-rata \textit{F-Measure} dengan nilai 95\%.
	\item Hasil analisis keakuratan algoritma SVM di setiap lokasi di dalam gedung A FMIPA USK yang diujicoba sebanyak 10 kali tiap label menunjukkan hasil keakuratan sebesar 86,67\%.
	\item Hasil pengujian fungsionalitas menggunakan metode Black Box menghasilkan semua fitur berhasil dijalankan dengan baik. Kemudian pengujian usability menggunakan metode UMUX mendapat nilai 93,84 sedangkan skor UMUX-lite adalah 84,13. Kedua skor tersebut berada pada kategori A+. Hal ini bermakna aplikasi dapat diterima dengan baik, dan memiliki skala B atau \textit{Excellence} berdasarkan grafik pada gambar\ref{grafikumuxlite}.
\end{enumerate}



\section{\uppercase{SARAN}}

Penelitian ini masih banyak kekurangan sehingga perlu dikembangkan agar menjadi lebih baik. Berikut adalah saran untuk penelitian ini:
\begin{enumerate}
	\item Aplikasi ini sebaiknya dikembangkan lebih luas lagi untuk seluruh area gedung kampus hingga satu Universitas agar orang-orang baru bisa paham dimana letak posisi suatu lokasi di dalam gedung dan estimasi jumlah di setiap area lokasi.
	\item Aplikasi LocaLization sebaiknya dibangun juga  versi iOS.
	\item Mencari metode klasifikasi yang lebih baik lagi untuk memprediksi lokasi pengguna.
	\item Sebaiknya ditambahkan lagi jumlah Beacon di setiap ruangan supaya meningkatkan keakuratan klasifikasi.

\end{enumerate}

\fancyhf{}
\fancyfoot[R]{\thepage}
\begin{comment}
\bibliography{daftar-pustaka}
\end{comment}
